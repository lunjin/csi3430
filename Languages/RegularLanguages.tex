\documentclass{beamer}

\usepackage{wasysym}
\usepackage{tikz}
\usetikzlibrary{automata,positioning,arrows.meta}
%\usepackage[graphics,tightpage,active,pdftex]{preview}
%\setlength{\PreviewBorder}{5pt}
%\PreviewEnvironment{tikzpicture}

\usepackage[linguistics]{forest}
\usepackage[many]{tcolorbox}
\usepackage{comment}
\usepackage{multicol}
\usepackage{listings}
\usepackage{prooftree}

\newif\ifframeinlbf
\frameinlbftrue
\makeatletter
\newcommand\listofframes{\@starttoc{lbf}}
\makeatother

\addtobeamertemplate{frametitle}{}{%
    \ifframeinlbf%
         \addtocontents{lbf}{%
            \protect\makebox[2em][l]{%
                \protect\usebeamercolor[fg]{structure}\insertframenumber\hfill%
            }%
            \protect\hyperlink{page.\insertpagenumber}\insertframetitle\par%
        }%
   \else\fi%
}

\AtBeginSection[]{
  \begin{frame}
  \vfill
  \centering
  \begin{beamercolorbox}[sep=8pt,center,shadow=true,rounded=true]{title}
    \usebeamerfont{title}\insertsectionhead\par%
  \end{beamercolorbox}
  \vfill
  \end{frame}
}

\newcommand{\myalert}[2]{%
    \tikz\node[coordinate](t#1){};\alert<#1>{#2}
}


\newcommand{\myexplain}[3]
{\begin{tcolorbox}
    [   tikznode boxed title,
        enhanced,
        arc=0mm,
        interior style={white},
        attach boxed title to top center= {yshift=-\tcboxedtitleheight/2},
        fonttitle=\bfseries,
        colbacktitle=white,coltitle=red,
        boxed title style={size=small,colframe=red,boxrule=0pt},
        title={\tikz\node[coordinate](s#1) {};#2}]
#3
\end{tcolorbox}
\begin{tikzpicture}[overlay]
\path<#1>[->, thick] (s#1) edge [bend left] (t#1);
\end{tikzpicture}
}

\newcommand{\explain}[2]
{\begin{tcolorbox}
    [   tikznode boxed title,
        enhanced,
        arc=0mm,
        interior style={white},
        attach boxed title to top center= {yshift=-\tcboxedtitleheight/2},
        fonttitle=\bfseries,
        colbacktitle=white,coltitle=red,
        boxed title style={size=small,colframe=red,boxrule=0pt},
        title={#1}]
#2
\end{tcolorbox}
}

\def\point{\begin{tikzpicture}
\fill[orange] (0,0) circle (2pt);
\end{tikzpicture}}

\setbeamertemplate{navigation symbols}{} %Remove navigation bar

\tikzstyle{every picture}+=[remember picture]

\usepackage{color}
\newcommand{\vet}[1]{\foreach \num in {#1}{\el{\num}}}
\newcommand{\el}[1]{\fbox{\parbox[c][2ex][c]{1.1em}{#1}}}
\newcommand{\id}[1]{\parbox[r][2ex][b]{1.7em}{\raggedleft #1}}
\newcommand{\bl}{\color{blue}}
\newcommand{\gr}{\color{gray}}
\newcommand{\re}{\color{red}}

\title{Regular and Non-Regular Languages and Decidability}
\author{Lunjin Lu}
\date{}


\begin{document}
\begin{comment}
\end{comment}
\frameinlbffalse
\begin{frame}[fragile,t,allowframebreaks]{List of Frames}
    \listofframes
\end{frame}
\frameinlbftrue


\frame{ \titlepage
}

\begin{frame}{Regular Langauges} 
\begin{theorem} 
The collection of regular langauges is closed under 
\begin{itemize} \small
  \item [1.] set union
  \item [2.] concatenation
  \item [3.] Kleene closure
  \item [4.] complement 
  \item [5.] set intersection
  \item [6.] set complement
\end{itemize}

\begin{proof}
(1), (2) and (3) follow from Kleene's theorem. Let $L$ be a regular langauge. 
There is a DFA $M=\langle Q,\Sigma,\delta,q_i, F\rangle$ such that $L=L(M)$. 
$\bar{M}=\langle Q,\Sigma,\delta,q_i, Q\setminus F\rangle$. Then 
$w\in L(\bar{M})$ iff $w\not\in L(M)$ implying $L(\bar{M})=\Sigma^*-L=\bar{L}$. 
Therefore, $\bar{L}$ is a regular language. Note that $L_1-L_2=L_1\cap \bar{L_2}$ and that (6) follows from (4) and (5). 
It remains to prove (5), which is done by constructing an accepting  automation for $L_1\cap L_2$ from accepting automations for   
$L_1$ and $L_2$. 
\end{proof}
\end{theorem}
\end{frame}

\begin{frame}{Non-regular Langauges}
There are many langauges that are not regular, including the following langauges over the alphabet $\Sigma=\{0,1\}$
\begin{itemize}
  \item $\{0^n 1^n\mid n\geq 0\}$;
  \item EQUAL
  \item PRIME
  \item PALINDROME
  \item SQUARE
  \item MOREONES
\end{itemize}
\end{frame}

\begin{frame}{Pumping Lemma}

\end{frame}

\begin{frame}{$\{0^n 1^n\mid n\geq 0\}$}

\end{frame}


\begin{frame}{EQUAL}

\end{frame}


\begin{frame}{PRIME}

\end{frame}


\begin{frame}{PALINDROME}

\end{frame}


\begin{frame}{SQUARE}

\end{frame}


\begin{frame}{MOREONES}

\end{frame}

\begin{frame}{Decidability}
An yes/no problem is solvable if there is an algorithm that provides 
a definitive answer the problem in a finite number of steps. 
\end{frame}

\begin{frame}{Decidability problems of Regular Langauges}
Some decifality questions about regular langagues are 
\begin{itemize}
  \item Are two FAs accepting the same languages?
  \item Are two regular expressionn equivalent?
  \item is the langauge of an FA empty?
  \item Is the langauges of an FA finite? 
\end{itemize}
\end{frame}

\begin{frame}{Emptiness} 
\begin{theorem}

\end{theorem}
\end{frame}

\begin{frame}{Finiteness}

\end{frame}

\end{document}