\documentclass{beamer}

\usepackage{wasysym}
\usepackage{tikz}
\usepackage[linguistics]{forest}
\usepackage[many]{tcolorbox}
\usepackage{comment}
\usepackage{multicol}
\usepackage{listings}
\usepackage{prooftree}

\newif\ifframeinlbf
\frameinlbftrue
\makeatletter
\newcommand\listofframes{\@starttoc{lbf}}
\makeatother

\addtobeamertemplate{frametitle}{}{%
    \ifframeinlbf%
         \addtocontents{lbf}{%
            \protect\makebox[2em][l]{%
                \protect\usebeamercolor[fg]{structure}\insertframenumber\hfill%
            }%
            \protect\hyperlink{page.\insertpagenumber}\insertframetitle\par%
        }%
   \else\fi%
}

\AtBeginSection[]{
  \begin{frame}
  \vfill
  \centering
  \begin{beamercolorbox}[sep=8pt,center,shadow=true,rounded=true]{title}
    \usebeamerfont{title}\insertsectionhead\par%
  \end{beamercolorbox}
  \vfill
  \end{frame}
}

\newcommand{\myalert}[2]{%
    \tikz\node[coordinate](t#1){};\alert<#1>{#2}
}


\newcommand{\myexplain}[3]
{\begin{tcolorbox}
    [   tikznode boxed title,
        enhanced,
        arc=0mm,
        interior style={white},
        attach boxed title to top center= {yshift=-\tcboxedtitleheight/2},
        fonttitle=\bfseries,
        colbacktitle=white,coltitle=red,
        boxed title style={size=small,colframe=red,boxrule=0pt},
        title={\tikz\node[coordinate](s#1) {};#2}]
#3
\end{tcolorbox}
\begin{tikzpicture}[overlay]
\path<#1>[->, thick] (s#1) edge [bend left] (t#1);
\end{tikzpicture}
}

\newcommand{\explain}[2]
{\begin{tcolorbox}
    [   tikznode boxed title,
        enhanced,
        arc=0mm,
        interior style={white},
        attach boxed title to top center= {yshift=-\tcboxedtitleheight/2},
        fonttitle=\bfseries,
        colbacktitle=white,coltitle=red,
        boxed title style={size=small,colframe=red,boxrule=0pt},
        title={#1}]
#2
\end{tcolorbox}
}

\def\point{\begin{tikzpicture}
\fill[orange] (0,0) circle (2pt);
\end{tikzpicture}}

\setbeamertemplate{navigation symbols}{} %Remove navigation bar

\tikzstyle{every picture}+=[remember picture]

\usepackage{color}
\newcommand{\vet}[1]{\foreach \num in {#1}{\el{\num}}}
\newcommand{\el}[1]{\fbox{\parbox[c][2ex][c]{1.1em}{#1}}}
\newcommand{\id}[1]{\parbox[r][2ex][b]{1.7em}{\raggedleft #1}}
\newcommand{\bl}{\color{blue}}
\newcommand{\gr}{\color{gray}}
\newcommand{\re}{\color{red}}

\title{Finite State Automata}
\author{Lunjin Lu}
\date{}


\begin{document}
\begin{comment}
\end{comment}
\frameinlbffalse
\begin{frame}[fragile,t,allowframebreaks]{List of Frames}
    \listofframes
\end{frame}
\frameinlbftrue


\frame{ \titlepage
}

\begin{frame}{Vending Machine}
The vending machine sells coffee. It accepts coins of 5\cent~ and 10\cent. It dispenses 
a can of coffee when it gets exact amount of 25 cents. 
It is stuck when the amount exceeds 25 cents.  

The machine has a number of states and two input symbols. 
Its construction is illustrated below. 

\begin{center}
\begin{tabular}{c|c|c}
state$\backslash$input       & 5\cent & 10\cent \\ \hline
q0     & q5     & q10 \\ \hline  
\end{tabular} 
\end{center} 
\end{frame}

\begin{frame}{Transition Diagram}

\end{frame} 

\begin{frame}{Language of Vending Machine} 
Which sequences of coins beget coffee? Starting at the initial state q0 and machines receives (consumes) 
coins one by one and changes its state according to the coin it gets. 
\begin{itemize}
  \item 5\cent 5\cent 10\cent 5\cent - coffee is dispensed
  \[q0 \longrightarrow^{5\cent} q5 \longrightarrow^{5\cent} q10 \longrightarrow^{10\cent} q20 \longrightarrow^{5\cent} q25  \]
  \item 5\cent 5\cent 10\cent 10\cent - machine is stuck
   \[q0 \longrightarrow^{5\cent} q5 \longrightarrow^{5\cent} q10 \longrightarrow^{10\cent} q20 \not\longrightarrow^{10\cent}  \]
\end{itemize}

\end{frame}

\begin{frame} {Stuckness and Trap States}
The vending machine is stuck on input string $\cent 5\cent 10\cent 10\cent$ 
because there is on transition from q20 labelled with 10\cent. 
A trap state  can be added so that the machine always make progress. The machine transitions to 
the trap state from a given state upon receiving a given input symbol
when it would otherwise get stuck from the state on reading the input symbol. 
The machine stays at the trap state upon reading any input symbol.  
The vending machine with an added trap state is diagramed below. 
\end{frame}

\begin{frame}{Deterministic Finite Automata}
A deterministic finite automaton is a 5-tuple $\langle Q, \Sigma, \delta, q_0, F\rangle$ where
\begin{itemize}
  \item Q is a finite set of states,
  \item $\Sigma$ is a finite set of symbols,
  \item $q_0\in Q$ ($q_0$ is called the initial state),
  \item $F\subseteq Q$ (states in F are called final states),
  \item $\delta:Q\times \Sigma\mapsto Q$ is called the transition function. 
\end{itemize}

\end{frame}

\begin{frame}{Traces/Runs}
Traces, the label of a trace, 
$p\leadsto^{w} q$ 

\end{frame}

\begin{frame}{Transition function on strings $\bar{\delta}$}

\end{frame}

\begin{frame}{Language Accepted by a DFA}

\end{frame}

\begin{frame}{Design of DFA - Example} 
Design a DFA that accepts $\{awa,bwb\mid w\in\{a,b\}^*\}$. 

\end{frame}

\begin{frame}{Design of DFA - Example}
Design a DFA that accepts $\{w\mid w\in \{0,1,2\}^*\wedge \mbox{Sum of digits in w is a multiple of 3}\}$. 

\end{frame}

\begin{frame}{Design of DFA - Example} 
Design a DFA that accepts EVEN-EVEN. 

\end{frame}


\begin{frame}{NFA}

\end{frame}

\begin{frame}{NFA Example}

\end{frame}

\begin{frame}{$\Lambda$-closure}
Inductive definition
\end{frame}

\begin{frame}{Transition function on strings $\bar{\delta}$}
$\bar{\delta}(q,w)$ is the set of states the machine 
may transition to starting at state $q$ after consuming input string $w$.
Those are the states $q'$ such that there is a trace from $q$ to $q'$ labelled with $w$. 
\[ \bar{\delta}(q,w) =\{q' \mid q\leadsto^{w} q'\}
\] 
Alternatively, \(\bar{\delta}(q,w) = \bar{\delta}(\Lambda\mbox{-closure}(\{q\}),w)\) where
\begin{eqnarray*}
\bar{\delta}(S,\Lambda) &=& S \\ 
\bar{\delta}(S,aw) &=&  \bar{\delta}(\Lambda\mbox{-closure}(\bigcup_{s\in S}\delta(s,a)),w)
\end{eqnarray*}
\end{frame}

\begin{frame}{Language Accepted by a NFA}
Let $M=$ be a NFA. 
\[L(M) = \{w \mid \exists q\in F. q_0\leadsto^{W} q\}\]
Alternatively, 

\[ L(M) = \{w \mid \bar{\delta}(q_0,w) \cap F\neq\emptyset\}\]

\end{frame}

\begin{frame}{Design of NFA}

\end{frame}

\end{document}