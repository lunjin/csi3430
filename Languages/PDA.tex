\documentclass{beamer}

\usepackage{wasysym}
\usepackage{tikz}
\usetikzlibrary{automata,positioning,arrows.meta}
%\usepackage[graphics,tightpage,active,pdftex]{preview}
%\setlength{\PreviewBorder}{5pt}
%\PreviewEnvironment{tikzpicture}

\usepackage[linguistics]{forest}
\usepackage[many]{tcolorbox}
\usepackage{comment}
\usepackage{multicol}
\usepackage{listings}
\usepackage{prooftree}

\newif\ifframeinlbf
\frameinlbftrue
\makeatletter
\newcommand\listofframes{\@starttoc{lbf}}
\makeatother

\addtobeamertemplate{frametitle}{}{%
    \ifframeinlbf%
         \addtocontents{lbf}{%
            \protect\makebox[2em][l]{%
                \protect\usebeamercolor[fg]{structure}\insertframenumber\hfill%
            }%
            \protect\hyperlink{page.\insertpagenumber}\insertframetitle\par%
        }%
   \else\fi%
}

\AtBeginSection[]{
  \begin{frame}
  \vfill
  \centering
  \begin{beamercolorbox}[sep=8pt,center,shadow=true,rounded=true]{title}
    \usebeamerfont{title}\insertsectionhead\par%
  \end{beamercolorbox}
  \vfill
  \end{frame}
}

\newcommand{\myalert}[2]{%
    \tikz\node[coordinate](t#1){};\alert<#1>{#2}
}


\newcommand{\myexplain}[3]
{\begin{tcolorbox}
    [   tikznode boxed title,
        enhanced,
        arc=0mm,
        interior style={white},
        attach boxed title to top center= {yshift=-\tcboxedtitleheight/2},
        fonttitle=\bfseries,
        colbacktitle=white,coltitle=red,
        boxed title style={size=small,colframe=red,boxrule=0pt},
        title={\tikz\node[coordinate](s#1) {};#2}]
#3
\end{tcolorbox}
\begin{tikzpicture}[overlay]
\path<#1>[->, thick] (s#1) edge [bend left] (t#1);
\end{tikzpicture}
}

\newcommand{\explain}[2]
{\begin{tcolorbox}
    [   tikznode boxed title,
        enhanced,
        arc=0mm,
        interior style={white},
        attach boxed title to top center= {yshift=-\tcboxedtitleheight/2},
        fonttitle=\bfseries,
        colbacktitle=white,coltitle=red,
        boxed title style={size=small,colframe=red,boxrule=0pt},
        title={#1}]
#2
\end{tcolorbox}
}

\def\point{\begin{tikzpicture}
\fill[orange] (0,0) circle (2pt);
\end{tikzpicture}}

\setbeamertemplate{navigation symbols}{} %Remove navigation bar

\tikzstyle{every picture}+=[remember picture]

\usepackage{color}
\newcommand{\vet}[1]{\foreach \num in {#1}{\el{\num}}}
\newcommand{\el}[1]{\fbox{\parbox[c][2ex][c]{1.1em}{#1}}}
\newcommand{\id}[1]{\parbox[r][2ex][b]{1.7em}{\raggedleft #1}}
\newcommand{\bl}{\color{blue}}
\newcommand{\gr}{\color{gray}}
\newcommand{\re}{\color{red}}

\title{Pushdown Automata}
\author{Lunjin Lu}
\date{}


\begin{document}
\begin{comment}
\end{comment}
\frameinlbffalse
\begin{frame}[fragile,t,allowframebreaks]{List of Frames}
    \listofframes
\end{frame}
\frameinlbftrue


\frame{ \titlepage
}


\begin{frame}{$PDA\subseteq CFL$}
\begin{lemma}
If a PDA M acceptes L then L is a CFL.
\end{lemma}
We assume that M satisfies the following properties. 
\begin{itemize}
  \item [(a)] It has a single accepting state $q_a$.
  \item [(b)] It empties the stack before accepting.
  \item [(c)] Each of its transitions either pushe a sympol or pops a symbol. It never does both or neither. 
\end{itemize}

Every PDA can be changed to another PDA that satisfies the above properties. 
\end{frame}

\begin{frame}{CFG construction} 
Let $M=\langle Q,\Sigma,\Gamma, \delta, q_0, q_a \rangle$. We construct the following CFG.
\[G=\langle \{A_{p,q}\mid p,q\in Q\}, \Sigma, A_{q_0,q_a}, R\] where R consists of 
\begin{itemize}
  \item \(A_{p,q}\rightarrow a A_{r,s} b\)
       for each $p,q,r,s\in Q$, 
        $a,b\in\Sigma\cup\{\Lambda\}$ such that 
        $\delta(p,a,\Lambda)\ni (r,t)$ and 
        $\delta(s,b,t)\ni (q,\Lambda)$ for $t\in\Gamma$.
         Note that there is no restriction on $p,q,r,s$ 
        other than they are states. 
  \item 
  \( A_{p,q} \rightarrow A_{p,r}A_{r,q}
  \) for each $p,q,r\in Q$. 
  \item  
   \( A_{q,q} \rightarrow \Lambda
   \) for each $q\in Q$.
\end{itemize}
\end{frame}

\begin{frame}{PDA $M_1$ for $\{0^n1 \mid n\geq 0\}$}

\end{frame}

\begin{frame}{CFG constructed from  $M_1$}
Production rules
\end{frame}

\begin{frame} { $L(G)=L(M)$}
The proof for $L(G)=L(M)$ is done by showing  $L(G)\subseteq L(M)$ and 
$L(M)\subseteq L(G)$ in two separate steps. 
\begin{lemma}
$L(G)\subseteq L(M)$. 

It is required to prove that 
$\langle q_0, w,\lambda\rangle \Longrightarrow^*_M
    \langle q_a,\lambda,\lambda\rangle$ 
whenever $A_{q_0,q_a}\Longrightarrow_G^* w$. We prove the following 
that is stronger than the required. 
\[\forall p,q,w. ( 
   [A_{p,q} \Longrightarrow_G^* w] \longrightarrow 
   [\langle p,w,\lambda\rangle \Longrightarrow_{M}^* 
    \langle q,\lambda,\lambda\rangle)
   ] )
 \] It is equivalent to prove that $\forall n. H(n)$ is true where  
\[ H(n) = \forall p,q,w. ( 
   [A_{p,q} \Longrightarrow_G^n w] \longrightarrow 
   [\langle p,w,\lambda\rangle \Longrightarrow_{M}^* 
    \langle q,\lambda,\lambda\rangle)
   ] )
\] 
\end{lemma}
\end{frame}

\end{document}